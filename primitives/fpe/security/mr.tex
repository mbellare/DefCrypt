%!TEX root = ../fpe.tex

\heading{Message Recovery.}
The other primary security notion considered significant for FPE schemes is that of message recovery (MR), mainly because of the practical reasons one would use FPE (i.e. to prevent adversaries from being able to decrypt real-world data).\

 MR security is formalized by two games: $\gMR{}$ and $\gMG{}$ (shown in \figref{fig-fpe-mr}), associated to FPE scheme $\FPE$ and adversary $\advA$. In both games, a sampler $\D$ chooses a key from $\Kg{\FPE}$, a nonce from $\NS{\FPE}$, and a target message from $\Dom{\FPE}$. In $\gMR{}$, $\Enc{\FPE}$ encrypts the target message, and the adversary $\advA$ is given the resulting ciphertext, the nonce used to encrypt it, and the security parameter. The adversary $\advA$ must then guess what the target message is. In attempting to do so, it may call an encryption oracle on pairs of messages and tweaks. \
 
 In the game $\gMG{}$, on the other hand, the simulator is not given the ciphertext of the target message, and instead of $\EncO$, its oracle is $\EqO$. $\EqO$ only confirms or denies whether a particular guess equals the target message.\
 
 The adversary's advantage in breaking the mr security of $\FPE$ is determined by how well it guesses target messages compared to the simulator when they make the same number of queries to their respective oracles. Formally, this advantage is defined as $\mrAdv{\FPE}{\advA} = \Pr[\gMR{\FPE}(\advA)]- \max_{\sampS}\Pr[\gMG{\FPE}(\sampS)]$, where $\max_{\sampS}\Pr[\gMG{\FPE}(\sampS)]$ is effectively the best a simulator could possibly do, i.e. the highest probability of guessing the target messages across all simulators.

\begin{figure} [t]
\begin{center}
\fbox
{
\begin{pchstack}
\procedure{Game $\gMR{\FPE}(\advA^{\EncO})$}
  {
    (\key, X^*, \nonce^*) \getsr  \D \\ 
	Y^* \gets \Enc{\FPE}(\key, \nonce^*, X^*)\\
    X \getsr \advA^{\EncO}(\secIn, \nonce^*, Y^*)  \\
    \pcreturn (X = X^*)
  }
    \pchspace
    
\procedure{Oracle $\EncO(\nonce,\msg)$}
  {
    Y\gets \Enc{\FPE}(\key,\nonce,\msg)  \\
    \pcreturn Y
  }
\end{pchstack}
}
\end{center}
\vspace{0mm}
\begin{center}
\fbox
{
\begin{pchstack}
\procedure{Game $\gMG{\FPE}(S^{\EqO})$}
  {
    (\key, X^*, \nonce^*) \getsr \D \\ 
    X \getsr \sampS^{\EqO}(\secIn, \nonce^*)  \\
    \pcreturn (X = X^*)
  }
    \pchspace
  
\procedure{Oracle $\EqO(\msg)$}
  {
    \pcreturn (M = X^*)
  }
\end{pchstack}
}
\end{center}
\vspace{-2ex}
\caption{ Games defining mr security of format-preserving encryption scheme $\FPE$.}
\label{fig-fpe-mr}
\hrulefill
\end{figure}

