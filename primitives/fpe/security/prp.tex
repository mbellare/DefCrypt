%!TEX root = ../fpe.tex

\heading{Pseudorandom Permutation.}
One of the primary security notions considered for FPE schemes is that of pseudorandom permutation (PRP) security.\

This is formalized by the game $\gPRP{}$ shown in \figref{fig-fpe-prp}, associated to FPE scheme $\FPE$ and adversary $\advA$. 
 The game initially samples a random challenge bit $b$, with $b=1$ indicating it is in ``real'' mode and $b=0$ that it is in ``random" mode.
 As per our conventions noted in \secref{sec-defs},  the tables $T$ and $R$ are intialized to $\bot$ everywhere.
 Now the adversary $\advA$ also has access to an evaluation oracle $\FnO$ that takes an input a nonce $\nonce$ and message $\msg$.
For a new $\nonce,\msg$ the oracle either returns a uniformly random element from $\Dom{\FF}$ which have not previously been returned (for $b=0$), or an evaluation under $\Enc{\FF}$ (for $b=1$).
That is, $\FnO$ is either a random permutation for each choice of $\nonce$ or $\Enc{\FPE}$ with a random secret key.
As a last step, the adversary outputs a bit $b'$ that can be viewed as a guess for $b$. 
 The advantage of adversary $\advA$ in breaking the prp security of  $\FPE$ is defined as $\prpAdv{\FPE}{\advA} = 2\Pr[\gPRP{\FPE}(\advA)]- 1$.

\begin{figure} [t]
\begin{center}
\fbox
{
\begin{pchstack}
\procedure{Game $\gPRP{\FPE}(\advA)$}
  {
    b \getsr \bits
    \key \getsr \Kg{\FPE}(\secIn)  \\
    b' \getsr \advA^{\RoRO}(\secIn)  \\
    \pcreturn (b' = b)
  }
  
    \pchspace
    
\procedure{Oracle $\FnO(\nonce,\msg)$}
  {
    \pcif (R[\nonce] =\bot) \pcthen  R[\nonce]\gets \emptyset\\
    \pcif (T[\nonce,\msg] =\bot) \pcthen  \\
    \t T[\nonce,\msg] \getsr \Dom{\FPE}\setminus R[\nonce] \\
    \t R[\nonce] \gets  R[\nonce]\cup\sett{T[\nonce,\msg]}\\
    \pcfi \\
    y_1\gets \Enc{\FPE}(\key,\nonce,\msg)  \\
    y_0\getsr T[\nonce,\msg] \\
   \pcreturn y_b
  }
\end{pchstack}
}
\end{center}
\vspace{-2ex}
\caption{ Game defining prp security of format-preserving encryption scheme $\FPE$.}
\label{fig-fpe-prp}
\hrulefill
\end{figure}

