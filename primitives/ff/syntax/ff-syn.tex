%!TEX root = ../ff.tex

\heading{Function families.}
A family of functions $\FF$ specifies PT algorithms $\Kg{\FF}$, $\Ev{\FF}$ and parameterized sets $\Dom{\FF}$, $\Rng{\FF}$.
Key generation algorithm $\Kg{\FF}$ takes a security parameter $\secIn$ to return a function key $\FKey$.
Deterministic evaluation algorithm $\Ev{\FF}$ takes $\secIn, \FKey$, and an input $x\in\Dom{\FF}(\secParam, \FKey)$ to return an output $y\in\Rng{\FF}(\secParam, \FKey)$, where $\Dom{\FF}(\secParam, \FKey)$ is the domain and $\Rng{\FF}(\secParam, \FKey)$ is the range of function $\Ev{\FF}(\secIn, \FKey, \cdot)$.

\heading{Puncturable function families.}
A \emph{puncturable} function family $\FF$ specifies additional PT algorithms $\PKg{\FF}$ and $\PEv{\FF}$.
Punctured key generation algorithm $\PKg{\FF}$ takes $\secIn$, a function key $\FKey\in[\Kg{\FF}(\secIn)]$ and a target input $x^*\in\Dom{\FF}(\secParam,\FKey)$ to return a ``punctured'' function key $\PFKey$.
Deterministic punctured evaluation algorithm $\PEv{\FF}$ takes $\secIn, \PFKey$ and an input $x\in\Dom{\FF}(\secParam,\FKey)$ to return an output $y\in\Rng{\FF}(\secParam, \FKey)$.
The correctness of punctured evaluation requires that $\PEv{\FF}(\secIn, \PFKey, x) = \Ev{\FF}(\secIn, \FKey, x)$ for
  all $\secParam\in\N$,
  all $\FKey\in[\Kg{\FF}(\secIn)]$,
  all $x^*\in\Dom{\FF}(\secParam, \FKey)$,
  all $\PFKey\in[\PKg{\FF}(\secIn, \FKey, x^*)]$
  and all $x\in\Dom{\FF}(\secParam, \FKey)\setminus\{x^*\}$.
Puncturable function families is a special case of constrained function families.
These notions were concurrently and independently introduced in \cite{AC:BonWat13,CCS:KPTZ13,PKC:BoyGolIva14}.

\heading{Trapdoor function families.} 
A \emph{trapdoor} function family $\FF$ specifies additional PT algorithms $\TKg{\FF}$ and $\TInv{\FF}$.
Trapdoored key generation algorithm $\TKg{\FF}$ takes $\secIn$ to return a pair of keys $(\FKey, \TFKey)$, where $\FKey$ is a function key and $\TFKey$ is a function trapdoor key.
Trapdoor inverse evaluation algorithm $\TInv{\FF}$ takes $\secIn$, $\TFKey$ and an output $y\in\Rng{\FF}(\secParam, \FKey)$ to return an input $x\in\Dom{\FF}(\secParam, \FKey)$.
For any $\secParam\in\N$ the key generation algorithm $\Kg{\FF}(\secIn)$ is defined to run $\TKg{\FF}(\secParam)$ in order to produce $(\FKey, \TFKey)$, and to return $\FKey$ as its output value.
The correctness of inverse evaluation requires that $\TInv{\FF}(\secIn, \TFKey, \Ev{\FF}(\secIn, \FKey, x)) = x$ for all $\secParam\in\N$, all $(\FKey, \TFKey)\in[\TKg{\FF}(\secIn)]$ and all $x\in\Dom{\FF}(\secParam, \FKey)$.

\heading{Extended syntax for function families.}
Let $\FF$ be a function family.
Let $\Keys{\FF}$ be the parameterized set such that $\Keys{\FF}(\secParam) = [\Kg{\FF}(\secIn)]$ for all $\secParam\in\N$.
We say that $\Keys{\FF}$ is the key space of $\FF$, $\Dom{\FF}$ is the domain of $\FF$, and $\Rng{\FF}$ is the range of $\FF$.
Let $\kl{\FF},\il{\FF},\ol{\FF}\colon\N\to\N$. 
We say that $\kl{\FF}$ is the key length of $\FF$, $\il{\FF}$ is the input length of $\FF$, and $\ol{\FF}$ is the output length of $\FF$ if $\Keys{\FF}(\secParam) = \bits^{\kl{\FF}(\secParam)}$, $\Dom{\FF}(\secParam,\FKey) = \bits^{\il{\FF}(\secParam)}$, and $\Rng{\FF}(\secParam, \FKey) = \bits^{\ol{\FF}(\secParam)}$ for all $\secParam\in\N$ and all $\FKey\in[\Kg{\FF}(\secParam)]$, respectively.
One can further specify the notion of the pre-image size of a function family, and use it to define what does it mean for a function family to be injective or to have a polynomial pre-image size (e.g.~see~\cite{AC:BelSteTes14}).

\heading{Block ciphers.}
Let $\FF$ be a function family with input length $\il{\FF}$ and output length $\ol{\FF}$, such that $\il{\FF}=\ol{\FF}$.
Function family $\FF$ is a block cipher if it specifies an additional PT algorithm $\Inv{\FF}$ as follows.
Inverse evaluation algorithm $\Inv{\FF}$ takes $\secIn$, $\FKey\in[\Kg{\FF}(\secIn)]$ and an output $y\in\Rng{\FF}(\secParam, \FKey)$ to return an input $x\in\Dom{\FF}(\secParam. \FKey)$.
The correctness of inverse evaluation requires that $\Inv{\FF}(\secIn, \FKey, \Ev{\FF}(\secIn, \FKey, x)) = x$ for all $\secParam\in\N$, all $\FKey\in[\Kg{\FF}(\secIn)]$, and all $x\in\Dom{\FF}(\secParam, \FKey)$.

\iffalse
\heading{Input- and output- sampleable function families.}
An input-sampleable function family $\FF$ specifies additional PT algorithm $\Is{\FF}$.
This algorithm takes $\secIn$, a function key $\FKey\in[\Kg{\FF}(\secParam)]$ to return an input $x\in\Dom{\FF}(\secParam,\FKey)$.
An output-sampleable function family $\FF$ specifies additional PT algorithm $\Os{\FF}$.
This algorithm takes $\secIn$, a function key $\FKey\in[\Kg{\FF}(\secParam)]$ to return an output $y\in\Rng{\FF}(\secParam,\FKey)$.


\heading{Function families.}
We say that $\FF$ is injective if the function $\FEv(\secIn,\FKey,\cdot)\Colon\FInpSet(\secParam)\to\FOutSet(\secParam)$ is injective for all $\secParam\in\N$ and $\FKey\in [\FKg(\secIn)]$.
	
I now give a slightly different proposed definition.\footnote{In the below I adopt that algorithms will always be considered to be PT unless specified otherwise.
Perhaps also the convention that any operation/algorithm applied to $\bot$ returns $\bot$ (in particular that the boolean $\bot=x$ equals $\bot$ for all $x$.)}

\begin{defn}
	An input-sampleable family of functions $\FF$ is a family of functions which additionally specifies a randomized algorithm $\Is{\FF}$.
	This algorithm outputs an $x\in\Dom{\FF}(\secParam,\key)$ when given security parameter $\secIn$ and key $\key\in\Keys{\FF}(\secParam)$ as inputs.  
\end{defn}

In this case we assume $\Kg{\FF}$ is the algorithm which samples $\key\in\bits^{\kl{\FF}(\secParam)}$ uniformly at random and $\Is{\FF}$ is the algorithm which samples $x\in\bits^{\il{\FF}(\secParam)}$ uniformly at random. 

\fi