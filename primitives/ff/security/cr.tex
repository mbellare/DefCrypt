%!TEX root = ../ff.tex

\begin{figure}[t]
\twoCols{0.22}{0.22}
{
\vspace{-1.0em} % TODO: temporary fix
\procedure{Game $\gCR{\FF,\advA}(\secParam)$}
  {
    \FKey\getsr\Kg{\FF}(\secIn)\\
	(x_0,x_1)\getsr\advA(\secIn,\FKey)\\
	y_0\gets\Ev{\FF}(\secIn,\FKey,x_0)\\
	y_1\gets\Ev{\FF}(\secIn,\FKey,x_1)\\
	\win_0\gets(x_0\neq x_1)\\
	\win_1\gets(y_0=y_1)\\
	\pcreturn (\win_0\land\win_1)
  }
}
{
\vspace{-1.0em} % TODO: temporary fix
\procedure{Game $\gTCR{\FF,\advA}(\secParam)$}
  {
    \FKey\getsr\Kg{\FF}(\secIn)\\
	x_0\getsr\Dom{\FF}(\secParam,\FKey)\\
	x_1\getsr\advA(\secIn,\FKey,x_0)\\
	y_0\gets\Ev{\FF}(\secIn,\FKey,x_0)\\
	y_1\gets\Ev{\FF}(\secIn,\FKey,x_1)\\
	\win_0\gets(x_0\neq x_1)\\
	\win_1\gets(y_0=y_1)\\
	\pcreturn (\win_0\land\win_1)
  }
}
\figvspace
\caption{Games defining
           collision-resistance of function family $\FF$,
		   and target collision-resistance of function family $\FF$.}
\label{fig-ff-cr}
\label{fig-ff-tcr}
\hrulefill
\end{figure}

\heading{Collision-resistant functions.}
Consider game $\gCR{\FF,\advA}(\secParam)$ of \figref{fig-ff-cr} associated to a function family $\FF$ and an adversary $\advA$.
The game runs $\Kg{\FF}$ to generate a function key $\FKey$, and gives it to the adversary $\advA$ as input.
The adversary wins the game if it can find two distinct inputs such that function $\Ev{\FF}(\secIn,\FKey,\cdot)$ maps them to the same output.
\begin{defn}
  Let $\FF$ be a function family.
  For all adversaries $\advA$ and all $\secParam\in\N$, let $\crAdv{\FF,\advA}{\secParam}=\Pr[\gCR{\FF,\advA}(\secParam)]$ be the advantage of $\advA$ in breaking the collision-resistance of $\FF$.
  We say that $\FF$ is collision-resistant if $\crAdv{\FF,\advA}{\cdot}$ is negligible for all PT adversaries $\advA$.
\end{defn}

\heading{Target collision-resistant functions.}
Consider game $\gTCR{\FF,\advA}(\secParam)$ of \figref{fig-ff-tcr} associated to a function family $\FF$ and an adversary $\advA$.
The game runs $\Kg{\FF}$ to generate a function key $\FKey$ and subsequently samples a uniformly random input value $x_0$ from the domain of function $\Ev{\FF}(\secIn, \FKey, \cdot)$.
The adversary gets $\FKey,x_0$ as input.
It wins the game if it can produce another input value $x_1$ such that $x_0$ and $x_1$ are distinct, and function $\Ev{\FF}(\secIn, \FKey, \cdot)$ maps inputs $x_0$, $x_1$ to the same output value.
\begin{defn}
  Let $\FF$ be a function family.
  $\FF$ is target collision-resistant if $\tcrAdv{\FF,\advA}{\cdot}$ is negligible for all PT adversaries $\advA$,
  where $\tcrAdv{\FF,\advA}{\secParam}=\Pr[\gTCR{\FF,\advA}(\secParam)]$ for all $\secParam\in\N$.
\end{defn}
Target collision-resistant hash functions were introduced by Naor and Yung~\cite{STOC:NaoYun89} under the name of Universal One-Way Hash Functions (UOWHF).
Bellare and Rogaway~\cite{C:BelRog97} redefined the corresponding security notion under the name of target collision-resistance.
We note that the standard definition of target collision-resistance is formalized in a different way from ours; it would require that adversary $\advA$ produces an input $x_0$ prior to seeing the function key.
Our definition of function families does not allow to do this, because for every $\secParam\in\N$ and for every $\FKey\in[\Kg{\FF}(\secIn)]$ the function $\Ev{\FF}(\secIn,\FKey,\cdot)$ is defined to take inputs from a set $\Dom{\FF}(\secParam,\FKey)$ that depends on $\FKey$.
Therefore, we provided an alternative definition for target collision-resistance.